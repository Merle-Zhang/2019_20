%ps1.solns.tex
%notes for the course Probability and Statistics COMS10011 
%taught at the University of Bristol
%2018_19 Conor Houghton conor.houghton@bristol.ac.uk

%To the extent possible under law, the author has dedicated all copyright 
%and related and neighboring rights to these notes to the public domain 
%worldwide. These notes are distributed without any warranty. 

\documentclass[11pt,a4paper]{scrartcl}
\typearea{12}
\usepackage{graphicx}
%\usepackage{pstricks}
\usepackage{listings}
\usepackage{color}
\lstset{language=C}
\usepackage{fancyhdr}
\pagestyle{fancy}
\lfoot{\texttt{github.com/COMS10011/2018\_19}}
\lhead{COMS10007 ps1.solns - Conor}
\begin{document}

\section*{Problem Sheet 1 - outline solutions}

\begin{enumerate}

\item In the poker hand two pair there are two pairs of cards with
  each card in the pair matched by value; the fifth card is a
  different value. What is the probability of two pairs when five
  cards are drawn randomly.\\ \\ \\ \textbf{Solution}: There are 13 choose two choices
  for the two values for the two pairs and for each pair there are
  four choose two possible cards. For the remaining card there are 11
  possible values and four possible suits. Thus, the number of
  possible pairs is
\begin{equation}
\left(\begin{array}{c}13\\2\end{array}\right)\left(\begin{array}{c}4\\2\end{array}\right)^2\times 44
=
\frac{13\times 12}{1\times 2}\times 36\times 44=123552
\end{equation}
and hence the probability is 123552/2598960=0.0475.

\item In a full house there is one pair and one
  triple, what is the probability of getting a full
  house?\\ \\ \\ \textbf{Solution}: For full house,
there are 13 possible values for the pair and 12 for the triple; including the choice of suits we have
\begin{equation}
13\times 12 \times \left(\begin{array}{c}4\\2\end{array}\right)\left(\begin{array}{c}4\\3\end{array}\right)=3744
\end{equation}
and the probability is 0.0014.

\item A student answers a multiple choice question with four options,
  one of which is correct. 80\% of students know the answer, 20\% of
  students guess and choose randomly. If a student gets the answer
  correct what is the chance they knew the
  answer.\\ \\ \\ \textbf{Solution}: Let $K$ be the event the student
  knows the right answer and $C$ is the event that the student chooses
  the correct answer. We want $P(K|C)$. By Bayes's rule
\begin{equation}
P(K|C)=\frac{P(C|K)P(K)}{P(C)}
\end{equation}
Now
\begin{equation}
  P(C)=P(C|K)P(K)+P(C|\bar{K})P(\bar{K})=0.8+0.25*0.2=0.85
\end{equation}
and hence
\begin{equation}
P(K|C)=\frac{0.8}{0.85}=0.94
\end{equation}

\item In the xkcd cartoon above, what is the chance the Bayesian will
  win his or her bet if the chance the sun has exploded is one in a
  million? In reality the chance is, of course, much less than one in
  a million! Show the answer to six decimal places.\\ \\ \\ \textbf{Solution}: Let $N$ be the event that the sun has exploded
  and $L$ be the event the machine says that the sun has
  exploded. Hence $P(L|N)=35/36$ whereas $P(L|\bar{N})=1/36$ and $P(N)=10^{-6}$. The Bayesian will
  win his or her bet is $P(\bar{N}|L)$:
\begin{equation}
P(\bar{N}|L)=\frac{P(L|\bar{N})P(\bar{N})}{P(L)}=\frac{1/36\times (1-10^{-6})}{1/36\times (1-10^{-6})+35/36\times 10^{-6}}=0.999965
\end{equation}

\item A three-sided dice is rolled three times. $X$ is the sum of the
  largest two values. Write down the probability distribution for
  $X$.\\ \\ \\ \textbf{Solution}: Well lets write down table and then
  explain where we got the numbers from
\begin{center}
\begin{tabular}{c|ccccc}
&2&3&4&5&6\\
\hline
$p_X$&1/27&1/9&7/27&1/3&7/27
\end{tabular}
\end{center}
So there are 27 possible outcomes of rolling the dice three times. To
get $X=2$ you need to roll 111, to get $X=3$ you can roll 112, 121 or
211. To get $X=4$ there are three permutations of 113 and three
permutations of 122, along with 222. To get $X=5$ there are six
permutations of 123 along with three permutations of 223. Finally to
rolls six there are three permutations of each of 133 and 233, along
with 333.

\end{enumerate}

\subsection*{Extra questions}

\begin{enumerate}


\item When it started in 1987 the Irish lottery has 36 numbers;
  participants paid 50 Irish pence to buy a combination of six
  different numbers; they would win if these numbers matched the six
  drawn. In the last week in May in 1992 a syndicate tried to buy all
  combinations of numbers. If they had succeeded how many numbers
  would they have bought?\\ \\ \\ \textbf{Solution}: Well this is just 36 choose six:
\begin{equation}
\left(\begin{array}{c}36\\6\end{array}\right)=1947792
\end{equation}
so they would've spend $973,896$ Irish pounds.


\item From a group of three undergraduates and five graduate students,
  four students are randomly selected to act as TAs. What is the
  chance there will be exactly two undergraduate
  TAs?\\ \\ \\ \textbf{Solution}: So this is another counting exercise,
  the total ways of selecting four out of eight is
\begin{equation}
\left(\begin{array}{c}8\\4\end{array}\right)=frac{8\times 7\times 6 \times 5}{1\times 2\times 3\times 4}
=70
\end{equation}
Now the number of ways of choosing two undergraduates out of three is
three and the number of ways of picking two graduates out of five is
10. Hence the answer is $3/7$.


\item Prove
\begin{equation}
\left(\begin{array}{c}n\\r\end{array}\right)=\left(\begin{array}{c}n\\n-r\end{array}\right)
\end{equation}
\\ \\ \\ \textbf{Solution}: This follows from the definition
\begin{equation}
\left(\begin{array}{c}n\\r\end{array}\right)=\frac{n!}{r!(n-r)!}
\end{equation}
which stays the same if you swap $r$ and $n-r$.

\item Two events $A$ and $B$ have probabilities $P(A)=0.2$, $P(B)=0.3$ and $P(A\cup B)=0.4$. Find
\begin{enumerate}
\item Find $P(A\cap B)$.
\item Find $P(\bar{A}\cap B)$.
\item Find $P(\bar{A}\cap \bar{B})$.
\item Find $P(A|B)$.
\end{enumerate}
\textbf{Solution}: So
\begin{equation}
P(A\cup B)=P(A)+P(B)-P(A\cap B)
\end{equation}
so 
\begin{equation}
P(A\cap B)=0.2+0.3-0.4=0.1
\end{equation}
If $P(A\cap B)=0.1$ then $P(\bar{A}\cap B)=P(B)-P(A\cap B)=0.2$. One the other hand
$P(\bar{A}\cap \bar{B})=1-P(A\cup B)=0.6$. Finally
\begin{equation}
P(A|B)=\frac{P(A\cap B)}{P(B)}=\frac{0.1}{0.3}=\frac{1}{3}
\end{equation}



\item One night in a bar in Las Vegas you meet a dodgy character who
  tells you that there are two types of slot machine in the Topicana,
  one that pays out 10\% of the time, the other 20\%. One sort of
  machine is blue, the other red. Unfortunately the dodgy character is
  too drunk to remember which is which. The next day you randomly
  select a red machine and put in a coin. You lose. Assuming the dodgy
  character was telling the truth, what is the chance the red machine
  is the one that pays out more. If you had won instead of losing,
  what would the chance be?\footnote{I stole this problem from
    \texttt{courses.smp.uq.edu.au/MATH3104/}}\\ \\ \\ \textbf{Solution}:
  So let $J$ be the event of winning and $R$ be the event that the red
  machine is the one that pays out more. We want
\begin{equation}
P(R|!J)=\frac{P(!J|R)P(R)}{P(!J)}
\end{equation}
where we are writing $!J$, not $J$, for $\bar{J}$. Now $P(!J|R)=0.8$
according to the DC and $P(R)=0.5$ because we chose between red and
blue randomly. Finally
\begin{equation}
P(!J)=P(!J|R)P(R)+P(!J|B)P(B)=0.8\times 0.5 + 0.9\times 0.5=0.85
\end{equation}
and hence
\begin{equation}
P(R|\bar{J})=\frac{0.8\times 0.5}{0.85}\approx0.47
\end{equation}
so, consider the prior $P(R)=0.5$ we haven't learned much from this lost coin. If we had won you'd have
\begin{equation}
P(R|J)=\frac{P(J|R)P(R)}{P(J)}=\frac{0.2\times 0.5}{0.15}\approx 0.67
\end{equation}
so you'd learn a lot more from a win, this makes sense since it is a rarer event.
\end{enumerate}

\end{document}

