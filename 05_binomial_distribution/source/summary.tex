
\ifind
\section*{Summary}
\else
\subsection*{5 Binomial distribution}
\fi


\begin{itemize}
\item In a \textbf{binomial experiment}  
\begin{enumerate}
\item There are $n$ identical trials.
\item Each trial has one of two outcomes, which we call success, $S$,
  and failure, $F$.
\item The trials are independent.
\item The random variable of interest, say $R$, is the total number of successes.
\end{enumerate}
\item In a binomial experiment, if $p$ is the chance of success for an individual trial, and $q=1-p$ is the chance of failure, then the probability of $r$ successes is given by
\begin{equation}
p_R(r)=\left(\begin{array}{c}n\\r\end{array}\right)p^rq^{n-r}
\end{equation}
\item The mean is $np$ and the variance is $npq$.
\item The mean is derived using a fancy trick involving differenciating
  \begin{equation}
  1=\sum_{r=0}^n \left(\begin{array}{c}n\\r\end{array}\right)p^rq^{n-r}=\sum_{r=0}^np(R=r)
  \end{equation}
  with respect to $p$.
\end{itemize}
